\documentclass[12pt]{article}
\usepackage{tikz}
\usepackage{float}
\usepackage[position=lflt]{floatflt}
\usetikzlibrary{arrows,%
                petri,%
                topaths}%	
\usepackage{tkz-berge}
\usepackage[T2A]{fontenc}
\usepackage[utf8]{inputenc}
\usepackage[english,russian]{babel}
\usepackage{amssymb,amsmath}
%\usepackage[usenames]{color}
\usepackage{graphics}
\usepackage{listings}
\usepackage{wrapfig}
\usepackage{lscape}
\usepackage{longtable}
\usepackage{hyperref}
%lstloadlanguages{C}
\usepackage{indentfirst}
\usepackage{multirow}
\usepackage[a4paper, left=20mm, right=20mm, top=15mm, bottom=15mm]{geometry}
%\usepackage{fancyhdr} %загрузим пакет
%\pagestyle{fancy} %применим колонтитул
%\fancyhead{} %очистим хидер на всякий случай
%\fancyhead[LE,RO]{\thepage} %номер страницы слева сверху на четных и справа на нечетных
\newcommand{\npl}{\nparallel}
\newcommand{\ephi}{\varphi}
\newcommand{\eps}{\varepsilon}
\newcommand{\tr}{\triangle}
\newcommand{\D}{\Delta}
\newcommand{\R}{\mathbb{R}}
\newcommand{\ibu}{\item}
\newcommand{\de}{\partial}
\newcommand{\aq}{\alpha}
\newcommand{\bq}{\beta}
\renewcommand{\lq}{\lambda}
\newcommand{\tint}{\int\limits}
\newcommand{\tlim}{\lim\limits}

\newcommand{\tsum}{\sum\limits}
\newcommand{\tsumn}{\sum\limits^n_{j =1}}
\newcommand{\grad}{\vec{\mbox{grad}}}
\newcommand{\I}{\vec{i}}
\newcommand{\J}{\vec{j}}
\newcommand{\K}{\vec{k}}
\newcommand{\vphi}{\varphi}
\newcommand{\Rr}{\mathbb{R}}
\newcommand{\Nn}{\mathbb{N}}
\newcommand{\Cc}{\mathbb{C}}
\newcommand{\Zz}{\mathbb{Z}}
\newcommand{\Dd}{\mathbb{D}}
\newcommand{\ldo}[1]{\mathcal{L}_n\left[{#1}\right]}


\newcommand{\ub}[2]{\underbrace{#1}_{#2}}

\newcommand{\liner}{\medskip\hrule\medskip}

\newcommand{\LRa}{\Leftrightarrow}
\newcommand{\Ra}{\Rightarrow}
\newcommand{\La}{\Leftarrow}

\renewcommand{\thesubsection}{\S\arabic{subsection}}
\renewcommand{\thesection}{ }
\newcommand{\method}[1]{\fbox{%
\parbox{17.2cm}{%
\fbox{%
\parbox{17cm}{%
{#1}
}%
}
}%
}
}


\begin{document}
Для работы собственного мини-инстапринтера требуются небольшие настройки. 
\begin{enumerate}
\item Если у вас всё ещё не установле python, его следует установить. Выбрать и скачать версию 2.x, {\bf не 3.x} 	\url{https://www.python.org/downloads/}
\item Скачать вспомогательные библиотеки: fpdf \url{https://code.google.com/p/pyfpdf/downloads/list}
\item На компьютерах с ОС Windows по умолчанию не стоит библиотека curl, ее так же следует установить \url{http://curl.haxx.se/download.html}
\item Скачиваем наш код \url{https://github.com/alexandrettio/instaprinter}. Скачать надо все папки и все файлы, сохраняя иерархию.
\item Исправляем ошибку в fpdf. Для этого заходим в папку fpdf, файл template.py, доходим до 60 строки и заменяем  
\begin{lstlisting}
value = value.encode("latin1","ignore") 
\end{lstlisting}
на
\begin{lstlisting}
pass
\end{lstlisting}
Производим установку через запуск setup.py в главной папке библиотеки
\item Открываем файл config.py. И заполняем поля нужными значениями. Сохраняем 
\item Открываем планировщик задач и создаем новый процесс, который должен запускаться каждые DELTA минут. DELTA указывается в config
\end{enumerate}
По возникшим вопросам можно писать мне на основную почту alexandrettio@gmail.com
\end{document}
